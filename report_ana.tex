\documentclass[12pt]{article}

\usepackage[a4paper,margin=2.5cm]{geometry}
\usepackage{amsmath,amssymb}
\usepackage{booktabs}
\usepackage{siunitx}
\usepackage{hyperref}

\sisetup{group-separator={,},group-minimum-digits=4}

\title{Risk Management Assignment -- Parts 5 \& 6: Hedging Strategy and Results}
\author{}
\date{}

\begin{document}
\maketitle

\section{Scope and objective}
This report presents the results for Parts 5 and 6 of the assignment, where the firm is exposed to copper price fluctuations through its material costs. The goal is to design a hedge and evaluate how hedging affects the distribution of EBIT under three operating cases: (a) fixed sales, and (b) sales correlated with the commodity with correlation parameters $\rho=0.74$ and $\rho=0.92$. Downside risk is measured by the $5\%$ Value-at-Risk (VaR$_{0.05}$) and the Expected Tail Loss (ETL$_{0.05}$). All risk numbers below are reported in USD millions, so a larger VaR/ETL corresponds to less severe downside outcomes for profitability.

\section{Methodology}
Hedges are constructed using the minimum-variance hedge ratio applied to first differences. Let $\Delta CM_t$ denote the change in the firm's total material costs (USD), and let $\Delta X_t$ denote the change in the hedge instrument price expressed in USD per ton. The hedge quantity (in tons) is estimated as
\[
h^\ast = \frac{\mathrm{Cov}(\Delta CM,\Delta X)}{\mathrm{Var}(\Delta X)}.
\]
Hedge P\&L is then computed as $\mathrm{PnL}_t = h^\ast \Delta X_t$ and added to each EBIT series. Economically, the firm is harmed by rising copper prices through higher input costs, so taking a \emph{long} position in the hedge instrument provides positive hedge P\&L when copper prices rise.

\section{Part 5: Hedging with copper futures and with a synthetic derivative}
\subsection{Unit conversion and contract sizing}
Copper futures (HG=F) are quoted in USD per pound. Prices are converted to USD per metric ton using
\[
1~\text{metric ton} = 2204.6226~\text{lb}, \qquad F^{\text{ton}}_t = F^{\text{lb}}_t \cdot 2204.6226.
\]
A standard contract covers $25{,}000$ lb, corresponding to approximately $11.3398$ metric tons.

\subsection{Part 5a: Hedge using actual copper futures}
The estimated hedge quantity using copper futures is
\[
h^\ast_{\text{fut}} = 32{,}193.71~\text{tons},
\qquad
N \approx \frac{32{,}193.71}{11.3398} \approx 2{,}839~\text{contracts}.
\]
Table~\ref{tab:p5_fut} compares VaR$_{0.05}$ and ETL$_{0.05}$ for unhedged and futures-hedged EBIT.

\begin{table}[h!]
\centering
\caption{Part 5A: EBIT risk (USD millions), unhedged vs.\ hedged with copper futures.}
\label{tab:p5_fut}
\begin{tabular}{llccc}
\toprule
Case & Strategy & $\alpha$ & VaR$_\alpha$ & ETL$_\alpha$ \\
\midrule
a) fixed sales & unhedged & 0.05 & 28.6 & 12.6 \\
b) $\rho=0.74$ & unhedged & 0.05 & -38.2 & -53.9 \\
b) $\rho=0.92$ & unhedged & 0.05 & 28.3 & 20.3 \\
a) fixed sales & hedged (copper futures) & 0.05 & 24.1 & 12.0 \\
b) $\rho=0.74$ & hedged (copper futures) & 0.05 & -36.8 & -61.7 \\
b) $\rho=0.92$ & hedged (copper futures) & 0.05 & 21.3 & 10.4 \\
\bottomrule
\end{tabular}
\end{table}

The futures hedge does not lead to uniform risk reduction in EBIT. In case (a) fixed sales, both VaR and ETL decrease after hedging, indicating worse left-tail profitability. In case (b) with $\rho=0.74$, VaR improves slightly (from $-38.2$ to $-36.8$), but ETL becomes more negative (from $-53.9$ to $-61.7$), meaning the most adverse outcomes in the far tail deteriorate. In case (b) with $\rho=0.92$, both VaR and ETL fall strongly, which is consistent with an operational natural hedge: when revenues co-move strongly with copper, the unhedged EBIT already benefits from offsetting movements, and a financial hedge can remove part of this beneficial co-movement.

\subsection{Part 5b: Hedge using a synthetic derivative with corr $\approx 0.75$ to costs}
A synthetic derivative instrument is constructed so that its changes have sample correlation 0.75 with $\Delta CM$. The estimated hedge quantity is
\[
h^\ast_{\text{deriv}} = 37{,}971.23~\text{tons},
\qquad
N \approx \frac{37{,}971.23}{11.3398} \approx 3{,}348~\text{contracts (futures-equivalent)}.
\]
Table~\ref{tab:p5_deriv} reports the corresponding VaR/ETL results.

\begin{table}[h!]
\centering
\caption{Part 5B: EBIT risk (USD millions), unhedged vs.\ hedged with synthetic derivative (corr $\approx 0.75$).}
\label{tab:p5_deriv}
\begin{tabular}{llccc}
\toprule
Case & Strategy & $\alpha$ & VaR$_\alpha$ & ETL$_\alpha$ \\
\midrule
a) fixed sales & unhedged & 0.05 & 28.6 & 12.6 \\
b) $\rho=0.74$ & unhedged & 0.05 & -38.2 & -53.9 \\
b) $\rho=0.92$ & unhedged & 0.05 & 28.3 & 20.3 \\
a) fixed sales & hedged (deriv corr$\sim$0.75) & 0.05 & 27.4 & 13.3 \\
b) $\rho=0.74$ & hedged (deriv corr$\sim$0.75) & 0.05 & -34.5 & -59.1 \\
b) $\rho=0.92$ & hedged (deriv corr$\sim$0.75) & 0.05 & 17.1 & 6.80 \\
\bottomrule
\end{tabular}
\end{table}

Compared to the futures hedge, the synthetic derivative hedge improves VaR more noticeably in the $\rho=0.74$ operating case (from $-38.2$ to $-34.5$), but ETL remains worse than unhedged. This indicates that correlation below one can reduce moderately adverse outcomes around the $5\%$ quantile while still leaving very severe tail outcomes largely unchanged or slightly worse. In the high revenue--commodity correlation case ($\rho=0.92$), hedging again lowers both VaR and ETL, reinforcing the interpretation that the firm already has a strong natural hedge through revenues.

\subsection{Implementation note: how the synthetic derivative was constructed}
To create a derivative series with a \emph{controlled} correlation to material-cost changes, the code starts from standardized cost changes and adds a noise term that is explicitly made orthogonal to those standardized changes. In practice, this is done by projecting random noise onto the cost-change direction and subtracting that projection, so the remaining noise has (approximately) zero sample correlation with the cost changes. The final derivative change is then formed as a weighted mixture of the standardized cost changes and the orthogonal noise, which guarantees the desired sample correlation level (e.g.\ 0.60, 0.75, 0.90) while keeping the variability comparable by scaling to a chosen target standard deviation.

\section{Part 6: Correlation sensitivity analysis (synthetic derivative)}
\subsection{Construction and hedge effectiveness metric}
Part 6 repeats the derivative hedge for correlations $\rho_{\text{deriv}}\in\{0.60,0.75,0.90\}$. Derivative changes are constructed so that their \emph{sample} correlation with $\Delta CM$ equals the target correlation, and the volatility is scaled to be comparable across correlation levels.

To evaluate hedge quality on the cost exposure alone, hedge effectiveness is computed as
\[
\mathrm{HE} = 1-\frac{\mathrm{Var}(\Delta CM-h^\ast\Delta D)}{\mathrm{Var}(\Delta CM)}.
\]
Table~\ref{tab:p6_eff} reports the resulting hedge ratios and hedge effectiveness.

\begin{table}[h!]
\centering
\caption{Part 6: Hedge effectiveness on material-cost changes.}
\label{tab:p6_eff}
\begin{tabular}{cccc}
\toprule
$\rho_{\text{deriv}}$ & Sample corr & $h^\ast$ (tons) & Hedge effectiveness \\
\midrule
0.60 & 0.60 & 30{,}377 & 0.360 \\
0.75 & 0.75 & 37{,}971 & 0.562 \\
0.90 & 0.90 & 45{,}565 & 0.810 \\
\bottomrule
\end{tabular}
\end{table}

As expected, hedge effectiveness increases strongly with correlation: moving from correlation 0.60 to 0.90 increases the fraction of cost-change variance removed by the hedge from $36\%$ to $81\%$.

\subsection{EBIT VaR/ETL across correlations}
Table~\ref{tab:p6_risk} reports EBIT VaR/ETL for each correlation level.

\begin{table}[h!]
\centering
\caption{Part 6: EBIT risk (USD millions) across derivative correlations.}
\label{tab:p6_risk}
\begin{tabular}{llccc}
\toprule
Case & Strategy & $\rho_{\text{deriv}}$ & VaR$_{0.05}$ & ETL$_{0.05}$ \\
\midrule
a) fixed sales & unhedged & -- & 28.6 & 12.6 \\
b) $\rho=0.74$ & unhedged & -- & -38.2 & -53.9 \\
b) $\rho=0.92$ & unhedged & -- & 28.3 & 20.3 \\
\midrule
a) fixed sales & hedged (deriv corr$\sim$0.6) & 0.60 & 25.9 & 16.1 \\
b) $\rho=0.74$ & hedged (deriv corr$\sim$0.6) & 0.60 & -36.1 & -57.5 \\
b) $\rho=0.92$ & hedged (deriv corr$\sim$0.6) & 0.60 & 24.1 & 13.6 \\
\midrule
a) fixed sales & hedged (deriv corr$\sim$0.75) & 0.75 & 27.4 & 13.3 \\
b) $\rho=0.74$ & hedged (deriv corr$\sim$0.75) & 0.75 & -34.5 & -59.1 \\
b) $\rho=0.92$ & hedged (deriv corr$\sim$0.75) & 0.75 & 17.1 & 6.80 \\
\midrule
a) fixed sales & hedged (deriv corr$\sim$0.9) & 0.90 & 23.0 & 10.1 \\
b) $\rho=0.74$ & hedged (deriv corr$\sim$0.9) & 0.90 & -38.3 & -67.7 \\
b) $\rho=0.92$ & hedged (deriv corr$\sim$0.9) & 0.90 & 12.6 & -6.08 \\
\bottomrule
\end{tabular}
\end{table}

The correlation sensitivity results show a clear distinction between hedging the cost exposure and improving EBIT tail risk. While hedge effectiveness on cost changes increases monotonically with correlation, the EBIT VaR/ETL outcomes do not improve monotonically. In particular, in the operating scenario with strong revenue--commodity correlation ($\rho=0.92$), aggressive cost hedging can significantly reduce VaR and even produce a negative ETL at $\rho_{\text{deriv}}=0.90$. This supports the interpretation that the firm has a natural hedge through revenues, and that a strong financial hedge can remove beneficial co-movement between revenues and costs.

\section{Conclusion}
The minimum-variance approach implies hedge sizes on the order of $3\times 10^4$ to $4.6\times 10^4$ tons, corresponding to thousands of standard futures contracts. Increasing correlation between the hedge instrument and cost changes strongly improves hedge effectiveness on the cost exposure. However, EBIT-based VaR/ETL results depend on the firm's operating structure. When revenues co-move strongly with copper prices, the firm benefits from an operational natural hedge, and additional financial hedging can worsen downside profitability outcomes. Therefore, a hedging policy aimed at reducing EBIT tail risk should be designed using the net exposure of EBIT (revenue and cost jointly), not only by minimizing variance of input costs.

\end{document}
