\documentclass[12pt]{article}

\usepackage[utf8]{inputenc}
\usepackage[english]{babel} 
\usepackage{graphicx}
\usepackage[hidelinks]{hyperref}
\usepackage{geometry}
\usepackage{titlesec}
\usepackage{csquotes}
\usepackage{biblatex}
\usepackage{array}
\usepackage{amsmath}
\usepackage{amssymb}
\usepackage{siunitx}
\usepackage{amsmath,amssymb}
\usepackage{booktabs}
\usepackage{hyperref}


\titlelabel{\thetitle.\quad}
\geometry{a4paper, margin=2.5cm}

\title{Portfolio Credit Risk models}
\author{Živa Artnak, Alja Dostal, Manca Kavčič, Anamarija Potokar}
\date{}

\begin{document}

\maketitle

% 1.-4. naloga

\section{Commodity selection and material cost modelling}
The company under consideration is a US-based non-financial firm whose production process is highly exposed to fluctuations in commodity prices. At current prices, the cost of the core input material represents approximately 69\% of the final product price, making commodity price risk the dominant source of operating risk. As a first step, an appropriate traded commodity must therefore be selected to proxy the dynamics of the company’s input costs.

\subsection{Choice of commodity and data}
Copper was selected as the reference commodity. This choice is economically justified as copper is a widely traded industrial metal, commonly used as an input in manufacturing processes, and exhibits substantial price volatility. Copper spot prices quoted in USD were obtained from a publicly available data source (index mundi) at monthly frequency, covering the period from December 2010 to June 2025. Using spot prices is consistent with the objective of capturing the underlying commodity price dynamics that affect the firm’s cost base.

Let $P_t^{Cu}$ denote the spot price of copper at time $t$.

\subsection{Mapping commodity prices to material costs}

At current prices, annual sales of the company amount to USD 920 million. Given that the core material represents 69\% of the sales price, the current annual material cost is
\[
CM_0 = 0.69 \times 920\,\text{million USD} = 634.8\, \text{million USD}.
\]

To link historical copper prices to the firm’s material costs, we assume a linear relationship between the commodity price and the cost of materials. Specifically, the material cost at time $t$ is modelled as
\[
CM_t = k \cdot P_t^{Cu},
\]
where the scaling factor $k$ represents the effective quantity of copper embedded in annual production. The parameter $k$ is calibrated such that the material cost at the current copper price matches the observed current material cost:
\[
k = \frac{CM_0}{P_0^{Cu}} = 64544.53,
\]
where $P_0^{Cu} = 9835.07$ is the most recent observed copper price.

This calibration ensures that the constructed material cost series is fully consistent with the company’s current profit and loss statement while preserving the historical dynamics of copper prices.

\subsection{Commodity price dynamics}

To analyse risk and to support later modelling steps, logarithmic returns of copper prices were computed as
\[
r_t^{Cu} = \ln(P_t^{Cu}) - \ln(P_{t-1}^{Cu}).
\]
Since the material cost is a linear transformation of the copper price, monthly logarithmic returns of copper prices are computed, as risk measurement and hedging effectiveness depend on price changes rather than price levels. Log returns are used due to their additive properties over time, which facilitates aggregation and historical simulation–based risk measures.

The resulting time series of material costs captures the historical variability of input costs faced by the firm and forms the basis for analysing operating risk, pricing pass-through mechanisms, and hedging strategies in the subsequent sections.

(TUKI POVEJTE A BI VE DALE NOTR TO ČASOVNO VRSTO OZ. TABELO SAM MA 176 VRSTIC)
\section{Sales modelling under alternative pass-through assumptions}

Having established a time series for the cost of the core input material based on copper prices, the next step is to model the company’s sales dynamics. The objective is to assess how different degrees of commodity price pass-through to customers affect operating performance and risk. Two conceptually distinct sales regimes are considered.

\subsection{Case (a): Fixed sales (no pass-through)}

In the first scenario, sales are assumed to be independent of commodity price movements. This setting reflects a situation in which the firm operates under fixed-price contracts with its customers, such that output prices are negotiated in advance and cannot be adjusted in response to changes in input costs.

Formally, annual sales are assumed to be constant at
\[
S_t = S_0 = 920\,\text{million USD} \quad \text{for all } t.
\]
In this case, sales exhibit no variability and are uncorrelated with copper price movements. All variability in operating performance therefore arises exclusively from fluctuations in material costs. This scenario serves as a benchmark representing maximum exposure to commodity price risk on the cost side.

\subsection{Case (b): Correlated sales (partial pass-through)}

In the second scenario, sales are allowed to co-move with commodity prices, reflecting the presence of partial price pass-through mechanisms. Economically, this corresponds to situations in which output prices can be adjusted in response to changes in input costs, for example through indexed contracts or pricing clauses linked to commodity markets.

Rather than modelling prices directly, sales dynamics are generated by constructing a time series of sales returns that is correlated with copper price returns. Let $r_t^{Cu}$ denote the logarithmic return of the copper price. Sales returns are modelled as
\[
r_t^{S} = \rho \, r_t^{Cu} + \sqrt{1 - \rho^2} \, \varepsilon_t,
\]
where $\rho$ is the target correlation coefficient between sales returns and commodity price returns, and $\varepsilon_t$ is an independent standard normal shock. This construction ensures that sales returns exhibit the desired correlation with copper prices while retaining an idiosyncratic component.

Two levels of pass-through are considered:
\begin{itemize}
  \item $\rho_1 = 0.74$, representing moderate pass-through to customers,
  \item $\rho_2 = 0.92$, representing strong pass-through and high pricing flexibility.
\end{itemize}

Given the simulated sales returns, the corresponding annual sales levels are obtained recursively as
\[
S_t = S_{t-1} \exp(r_t^{S}),
\]
with initial sales set equal to the current annual sales level, $S_0 = 920$ million USD. As a result, sales are no longer constant and may deviate from the initial level depending on the realised commodity price dynamics.

In contrast to the fixed-sales case, sales now fluctuate over time and respond positively to increases in copper prices. Higher values of the correlation parameter $\rho$ lead to stronger co-movement between sales and commodity prices, resulting in greater variability of sales levels.

\subsection{Empirical Properties of Simulated Sales}

The constructed sales return series, display several economically intuitive properties. First, sales returns are centred around zero, indicating the absence of deterministic growth or decline in sales. Second, volatility increases with the degree of pass-through, reflecting stronger sensitivity of revenues to commodity price movements. Finally, the correlation structure ensures that periods of rising copper prices are, on average, associated with increasing sales revenues, providing a natural hedge against rising input costs.

This modelling approach deliberately reverses the usual empirical estimation framework. Instead of estimating pass-through from observed sales data, sales are generated directly from the commodity price process. This allows for a controlled analysis of how different degrees of pass-through affect operating profitability and risk, which is the primary objective of the present study.

The resulting sales series form the basis for the EBIT simulations and risk analysis presented in the following section.

\section{Simulation of EBIT and comparison with the point estimate}
\label{sec:3}

Based on the simulated time series of material costs and sales described in the previous sections, the company’s operating performance is analysed by constructing a time series of Earnings Before Interest and Taxes (EBIT). At this stage, the financial part of the profit and loss statement is deliberately excluded in order to focus exclusively on operating risk arising from commodity price fluctuations and pricing pass-through mechanisms.

\subsection{EBIT definition and cost structure}

EBIT at time $t$ is defined as
\[
\text{EBIT}_t = S_t - CM_t - OC_t - DL_t - OH,
\]
where:
\begin{itemize}
  \item $S_t$ denotes annual sales,
  \item $CM_t$ denotes the cost of the core material,
  \item $OC_t = 0.06 \, S_t$ represents other variable material costs,
  \item $DL_t = 0.14 \, S_t$ represents direct labour costs,
  \item $OH = 77$ million USD denotes annual overhead costs.
\end{itemize}

Material costs are modelled as a linear function of the copper price, while other cost components are proportional to sales or fixed. This structure reflects the firm’s strong exposure to commodity price risk, as nearly 70\% of the sales price is driven by the core input material.

\subsection{Point estimate of EBIT}

Using current prices and the given profit and loss data, the point estimate of EBIT is
\[
\text{EBIT}_0 = 24.2 \text{ million USD}.
\]
This value reflects the company’s operating profitability at the most recent observed copper price and serves as a benchmark against which simulated EBIT outcomes are compared.

\subsection{Simulated EBIT under alternative sales regimes}

EBIT is simulated for each historical observation using the corresponding material cost and sales levels under the three sales regimes introduced previously:
\begin{itemize}
  \item fixed sales (no pass-through),
  \item partial pass-through with $\rho_1 = 0.74$,
  \item strong pass-through with $\rho_2 = 0.92$.
\end{itemize}

For each scenario, a time series of annual EBIT levels is obtained. Table~\ref{tab:ebit_results} summarises the average simulated EBIT and the EBIT corresponding to the most recent observation (current prices).

\begin{table}[h!]
\centering
\caption{Simulated EBIT under Alternative Sales Regimes}
\label{tab:ebit_results}
\begin{tabular}{lcc}
\hline
\textbf{Scenario} & \textbf{Average EBIT} & \textbf{Current EBIT} \\
 & (USD million) & (USD million) \\
\hline
Fixed sales (no pass-through) & 183.2 & 24.2 \\
Partial pass-through ($\rho_1 = 0.74$) & 117.5 & 7.3 \\
Strong pass-through ($\rho_2 = 0.92$) & 96.8 & 69.4 \\
\hline
\end{tabular}
\end{table}

The average simulated EBIT differs substantially across scenarios. Under fixed sales, average EBIT is high relative to the current point estimate, reflecting the fact that historical copper prices were often lower than current levels. However, this scenario also exhibits significant variability in EBIT, as all commodity price fluctuations directly affect costs.

\subsection{Interpretation}

Several important insights emerge from the EBIT simulations. First, the current EBIT level is substantially lower than the historical average under all scenarios, indicating that the firm is currently operating in an unfavourable commodity price environment. This is particularly evident in the fixed-sales case, where elevated copper prices significantly compress margins.

Second, sales pass-through acts as a natural hedge against commodity price risk. Strong pass-through ($\rho_2 = 0.92$) substantially improves current EBIT relative to the fixed-sales case, raising EBIT from USD 24.2 million to USD 69.4 million at current prices. However, partial pass-through ($\rho_1 = 0.74$) proves insufficient to offset high input costs and results in even weaker current EBIT.

Finally, increasing pass-through compresses the EBIT distribution, while strong pass-through improves downside protection, it also reduces average EBIT by limiting upside potential when commodity prices are low. This trade-off highlights the importance of evaluating operating performance not only in terms of expected profitability but also in terms of downside risk, which is analysed formally in the next section using Value-at-Risk and Expected Tail Loss measures.

\section{Value-at-Risk, Expected Tail Loss and management implications}
\label{sec:4}

While average EBIT provides useful information about expected operating performance, it does not adequately capture downside risk arising from adverse commodity price movements. To assess the firm’s exposure to extreme but plausible outcomes, this section evaluates the risk of EBIT using Value-at-Risk (VaR) and Expected Tail Loss (ETL).

\subsection{Risk measures}

Value-at-Risk at confidence level $\alpha = 5\%$ is defined as the lower $\alpha$-quantile of the EBIT distribution:
\[
\text{VaR}_{5\%} = \inf \{ x : \mathbb{P}(\text{EBIT} \leq x) \geq 5\% \}.
\]
It represents the EBIT level that will not be exceeded with 95\% confidence.

Expected Tail Loss (ETL), also referred to as Expected shortfall, is defined as the average EBIT conditional on being in the worst 5\% of outcomes:
\[
\text{ETL}_{5\%} = \mathbb{E}[\text{EBIT} \mid \text{EBIT} \leq \text{VaR}_{5\%}].
\]
ETL provides a more conservative measure of risk by capturing the severity of losses beyond the VaR threshold.

Both measures are computed directly from the simulated EBIT distributions obtained in the previous section.

\subsection{Empirical results}

Table~\ref{tab:var_etl} reports the VaR and ETL at the 5\% level for all three sales regimes.

\begin{table}[h!]
\centering
\caption{VaR and ETL at the 5\% level}
\label{tab:var_etl}
\begin{tabular}{lcc}
\hline
\textbf{Scenario} & \textbf{VaR (5\%)} & \textbf{ETL (5\%)} \\
 & (USD million) & (USD million) \\
\hline
Fixed sales (no pass-through) & 29.5 & 13.9 \\
Partial pass-through ($\rho_1 = 0.74$) & -34.8 & -52.2 \\
Strong pass-through ($\rho_2 = 0.92$) & 29.2 & 21.3 \\
\hline
\end{tabular}
\end{table}

Positive values indicate that EBIT remains positive even in the worst 5\% of outcomes, while negative values correspond to operating losses in adverse states.

\subsection{Interpretation}

The results highlight substantial differences in downside risk across sales regimes. Under fixed sales, the firm remains profitable even in the lower tail of the EBIT distribution, although margins are significantly compressed. Introducing partial pass-through with $\rho_1 = 0.74$ markedly worsens downside risk, both VaR and ETL become negative, indicating that the firm would incur operating losses in severe but plausible scenarios. This outcome reflects the fact that moderate pass-through is insufficient to offset high commodity input costs, while simultaneously reducing upside potential when commodity prices are low.

In contrast, strong pass-through ($\rho_2 = 0.92$) significantly improves downside protection. Both VaR and ETL are positive and exceed those observed in the fixed-sales case, demonstrating that high pricing flexibility acts as an effective natural hedge against commodity price risk.

\subsection{Recommendations to the management board}

Several clear recommendations emerge from the analysis. First, reliance on partial pass-through pricing mechanisms should be avoided, as they may increase downside risk and expose the firm to operating losses in adverse commodity price environments. Second, where market conditions permit, the firm should aim to implement pricing structures with strong commodity price indexation, as high pass-through substantially stabilizes EBIT and improves tail outcomes. Third, even under strong pass-through, residual risk remains. Therefore, pricing policy should be complemented by a formal commodity hedging program using financial derivatives. Such a program should be designed with the explicit objective of reducing VaR and ETL rather than maximizing expected profits. Aligning pricing flexibility with financial hedging would allow the firm to achieve a more resilient operating performance and improve planning certainty.

Overall, the results demonstrate that effective management of commodity risk requires a coordinated approach combining commercial pricing decisions with financial risk management instruments.


% 5. in 6. naloga

\sisetup{group-separator={,},group-minimum-digits=4}


\section{Hedging setup and methodology}

\subsection{Scope and objective}
This report presents the results for Parts 5 and 6 of the assignment, where the firm is exposed to copper price fluctuations through its material costs. The goal is to design a hedge and evaluate how hedging affects the distribution of EBIT under three operating cases: (a) fixed sales, and (b) sales correlated with the commodity with correlation parameters $\rho=0.74$ and $\rho=0.92$. Downside risk is measured by the $5\%$ Value-at-Risk (VaR$_{0.05}$) and the Expected Tail Loss (ETL$_{0.05}$). All risk numbers below are reported in USD millions, so a larger VaR/ETL corresponds to less severe downside outcomes for profitability.

\subsection{Hedge construction}
Hedges are constructed using the minimum-variance hedge ratio applied to first differences. Let $\Delta CM_t$ denote the change in the firm's total material costs (USD), and let $\Delta X_t$ denote the change in the hedge instrument price expressed in USD per ton. The hedge quantity (in tons) is estimated as
\[
h^\ast = \frac{\mathrm{Cov}(\Delta CM,\Delta X)}{\mathrm{Var}(\Delta X)}.
\]
Hedge P\&L is then computed as $\mathrm{PnL}_t = h^\ast \Delta X_t$ and added to each EBIT series. Economically, the firm is harmed by rising copper prices through higher input costs, so taking a \emph{long} position in the hedge instrument provides positive hedge P\&L when copper prices rise.

\subsection{Unit conversion and contract sizing}
Copper futures (HG=F) are quoted in USD per pound. Prices are converted to USD per metric ton using
\[
1~\text{metric ton} = 2204.6226~\text{lb}, \qquad F^{\text{ton}}_t = F^{\text{lb}}_t \cdot 2204.6226.
\]
A standard contract covers $25{,}000$ lb, corresponding to approximately $11.3398$ metric tons.

\subsection{Implementation note: how the synthetic derivative was constructed}
To create a derivative series with a \emph{controlled} correlation to material-cost changes, the code starts from standardized cost changes and adds a noise term that is explicitly made orthogonal to those standardized changes. In practice, this is done by projecting random noise onto the cost-change direction and subtracting that projection, so the remaining noise has (approximately) zero sample correlation with the cost changes. The final derivative change is then formed as a weighted mixture of the standardized cost changes and the orthogonal noise, which guarantees the desired sample correlation level (e.g.\ 0.60, 0.75, 0.90) while keeping the variability comparable by scaling to a chosen target standard deviation.

\section{Results for Parts 5 and 6}

\subsection{Hedge using actual copper futures}
The estimated hedge quantity using copper futures is
\[
h^\ast_{\text{fut}} = 32{,}193.71~\text{tons},
\qquad
N \approx \frac{32{,}193.71}{11.3398} \approx 2{,}839~\text{contracts}.
\]
Table~\ref{tab:p5_fut} compares VaR$_{0.05}$ and ETL$_{0.05}$ for unhedged and futures-hedged EBIT.

\begin{table}[h!]
\centering
\caption{Part 5A: EBIT risk (USD millions), unhedged vs.\ hedged with copper futures.}
\label{tab:p5_fut}
\begin{tabular}{llccc}
\toprule
Case & Strategy & $\alpha$ & VaR$_\alpha$ & ETL$_\alpha$ \\
\midrule
a) fixed sales & unhedged & 0.05 & 28.6 & 12.6 \\
b) $\rho=0.74$ & unhedged & 0.05 & -38.2 & -53.9 \\
b) $\rho=0.92$ & unhedged & 0.05 & 28.3 & 20.3 \\
a) fixed sales & hedged (copper futures) & 0.05 & 24.1 & 12.0 \\
b) $\rho=0.74$ & hedged (copper futures) & 0.05 & -36.8 & -61.7 \\
b) $\rho=0.92$ & hedged (copper futures) & 0.05 & 21.3 & 10.4 \\
\bottomrule
\end{tabular}
\end{table}

The futures hedge does not lead to uniform risk reduction in EBIT. In case (a) fixed sales, both VaR and ETL decrease after hedging, indicating worse left-tail profitability. In case (b) with $\rho=0.74$, VaR improves slightly (from $-38.2$ to $-36.8$), but ETL becomes more negative (from $-53.9$ to $-61.7$), meaning the most adverse outcomes in the far tail deteriorate. In case (b) with $\rho=0.92$, both VaR and ETL fall strongly, which is consistent with an operational natural hedge: when revenues co-move strongly with copper, the unhedged EBIT already benefits from offsetting movements, and a financial hedge can remove part of this beneficial co-movement.

\subsection{Hedge using a synthetic derivative with corr $\approx 0.75$ to costs}
A synthetic derivative instrument is constructed so that its changes have sample correlation 0.75 with $\Delta CM$. The estimated hedge quantity is
\[
h^\ast_{\text{deriv}} = 37{,}971.23~\text{tons},
\qquad
N \approx \frac{37{,}971.23}{11.3398} \approx 3{,}348~\text{contracts (futures-equivalent)}.
\]
Table~\ref{tab:p5_deriv} reports the corresponding VaR/ETL results.

\begin{table}[h!]
\centering
\caption{Part 5B: EBIT risk (USD millions), unhedged vs.\ hedged with synthetic derivative (corr $\approx 0.75$).}
\label{tab:p5_deriv}
\begin{tabular}{llccc}
\toprule
Case & Strategy & $\alpha$ & VaR$_\alpha$ & ETL$_\alpha$ \\
\midrule
a) fixed sales & unhedged & 0.05 & 28.6 & 12.6 \\
b) $\rho=0.74$ & unhedged & 0.05 & -38.2 & -53.9 \\
b) $\rho=0.92$ & unhedged & 0.05 & 28.3 & 20.3 \\
a) fixed sales & hedged (deriv corr$\sim$0.75) & 0.05 & 27.4 & 13.3 \\
b) $\rho=0.74$ & hedged (deriv corr$\sim$0.75) & 0.05 & -34.5 & -59.1 \\
b) $\rho=0.92$ & hedged (deriv corr$\sim$0.75) & 0.05 & 17.1 & 6.80 \\
\bottomrule
\end{tabular}
\end{table}

Compared to the futures hedge, the synthetic derivative hedge improves VaR more noticeably in the $\rho=0.74$ operating case (from $-38.2$ to $-34.5$), but ETL remains worse than unhedged. This indicates that correlation below one can reduce moderately adverse outcomes around the $5\%$ quantile while still leaving very severe tail outcomes largely unchanged or slightly worse. In the high revenue--commodity correlation case ($\rho=0.92$), hedging again lowers both VaR and ETL, reinforcing the interpretation that the firm already has a strong natural hedge through revenues.

\subsection{Correlation sensitivity analysis (synthetic derivative)}
Part 6 repeats the derivative hedge for correlations $\rho_{\text{deriv}}\in\{0.60,0.75,0.90\}$. Derivative changes are constructed so that their \emph{sample} correlation with $\Delta CM$ equals the target correlation, and the volatility is scaled to be comparable across correlation levels.

To evaluate hedge quality on the cost exposure alone, hedge effectiveness is computed as
\[
\mathrm{HE} = 1-\frac{\mathrm{Var}(\Delta CM-h^\ast\Delta D)}{\mathrm{Var}(\Delta CM)}.
\]
Table~\ref{tab:p6_eff} reports the resulting hedge ratios and hedge effectiveness.

\begin{table}[h!]
\centering
\caption{Part 6: Hedge effectiveness on material-cost changes.}
\label{tab:p6_eff}
\begin{tabular}{cccc}
\toprule
$\rho_{\text{deriv}}$ & Sample corr & $h^\ast$ (tons) & Hedge effectiveness \\
\midrule
0.60 & 0.60 & 30{,}377 & 0.360 \\
0.75 & 0.75 & 37{,}971 & 0.562 \\
0.90 & 0.90 & 45{,}565 & 0.810 \\
\bottomrule
\end{tabular}
\end{table}

As expected, hedge effectiveness increases strongly with correlation: moving from correlation 0.60 to 0.90 increases the fraction of cost-change variance removed by the hedge from $36\%$ to $81\%$.

\subsection{EBIT VaR/ETL across correlations}
Table~\ref{tab:p6_risk} reports EBIT VaR/ETL for each correlation level.

\begin{table}[h!]
\centering
\caption{Part 6: EBIT risk (USD millions) across derivative correlations.}
\label{tab:p6_risk}
\begin{tabular}{llccc}
\toprule
Case & Strategy & $\rho_{\text{deriv}}$ & VaR$_{0.05}$ & ETL$_{0.05}$ \\
\midrule
a) fixed sales & unhedged & -- & 28.6 & 12.6 \\
b) $\rho=0.74$ & unhedged & -- & -38.2 & -53.9 \\
b) $\rho=0.92$ & unhedged & -- & 28.3 & 20.3 \\
\midrule
a) fixed sales & hedged (deriv corr$\sim$0.6) & 0.60 & 25.9 & 16.1 \\
b) $\rho=0.74$ & hedged (deriv corr$\sim$0.6) & 0.60 & -36.1 & -57.5 \\
b) $\rho=0.92$ & hedged (deriv corr$\sim$0.6) & 0.60 & 24.1 & 13.6 \\
\midrule
a) fixed sales & hedged (deriv corr$\sim$0.75) & 0.75 & 27.4 & 13.3 \\
b) $\rho=0.74$ & hedged (deriv corr$\sim$0.75) & 0.75 & -34.5 & -59.1 \\
b) $\rho=0.92$ & hedged (deriv corr$\sim$0.75) & 0.75 & 17.1 & 6.80 \\
\midrule
a) fixed sales & hedged (deriv corr$\sim$0.9) & 0.90 & 23.0 & 10.1 \\
b) $\rho=0.74$ & hedged (deriv corr$\sim$0.9) & 0.90 & -38.3 & -67.7 \\
b) $\rho=0.92$ & hedged (deriv corr$\sim$0.9) & 0.90 & 12.6 & -6.08 \\
\bottomrule
\end{tabular}
\end{table}

The correlation sensitivity results show a clear distinction between hedging the cost exposure and improving EBIT tail risk. While hedge effectiveness on cost changes increases monotonically with correlation, the EBIT VaR/ETL outcomes do not improve monotonically. In particular, in the operating scenario with strong revenue--commodity correlation ($\rho=0.92$), aggressive cost hedging can significantly reduce VaR and even produce a negative ETL at $\rho_{\text{deriv}}=0.90$. This supports the interpretation that the firm has a natural hedge through revenues, and that a strong financial hedge can remove beneficial co-movement between revenues and costs.

\subsection{Conclusion}
The minimum-variance approach implies hedge sizes on the order of $3\times 10^4$ to $4.6\times 10^4$ tons, corresponding to thousands of standard futures contracts. Increasing correlation between the hedge instrument and cost changes strongly improves hedge effectiveness on the cost exposure. However, EBIT-based VaR/ETL results depend on the firm's operating structure. When revenues co-move strongly with copper prices, the firm benefits from an operational natural hedge, and additional financial hedging can worsen downside profitability outcomes. Therefore, a hedging policy aimed at reducing EBIT tail risk should be designed using the net exposure of EBIT (revenue and cost jointly), not only by minimizing variance of input costs.









% 7. in 8. naloga

\section{Transition to a EU-based company}
\label{sec:7}

In this section, we reconsider the risk profile under the assumption that the company is based in the European Union. The functional currency is now the Euro (EUR), while the primary raw material (Copper) remains quoted in US Dollars (USD).

The company’s financial parameters were adjusted to:
\begin{itemize}
    \item \textbf{Initial Sales ($S_0$):} 920,000,000 EUR
    \item \textbf{Overhead Costs (OH):} 77,000,000 EUR
    \item \textbf{Debt:} 540,000,000 EUR
\end{itemize}

Copper prices in EUR are subject to EUR/USD exchange rate, which was obtained from publicly available data source (Yahoo Finance).
\[ P_t^{Cu,\text{EUR}} = P_t^{Cu} \cdot FX_t^{\text{USD/EUR}} \]

From here on, the calculations are the same as in sections 1-6.

\subsection{Results for EU-based company}

Current copper price is $P_0^{Cu,\text{EUR}} = 8643.65$ with current FX rate $FX_0^{\text{EUR/USD}}=1.137838$.
Factor $k$ is $73441.2$.\\

As before (in section \ref{sec:3}), we present EBIT for different scenarios in table \ref{tab:ebit_results_eur}.
There is a noticable difference in EBIT values when comparing them to those in section \ref{sec:3}.
That is because od added FX risk. A weakening Euro acts like an increase in the price of copper.
Because of the nature of the model (that material represents $69\%$ of the price of final products), every time the Euro weakened, sales prices were raised to compensate.
Because fixed overheads remained constant at 77M EUR, the firm experienced significant margin expansion in nominal terms.

\begin{table}[h!]
\centering
\caption{Simulated EBIT under Alternative Sales Regimes}
\label{tab:ebit_results_eur}
\begin{tabular}{lcc}
\hline
\textbf{Scenario} & \textbf{Average EBIT} & \textbf{Current EBIT} \\
 & (EUR million) & (EUR million) \\
\hline
Fixed sales (no pass-through) & 197.1 & 24.2 \\
Partial pass-through ($\rho_1 = 0.74$) & 221.1 & 124.9 \\
Strong pass-through ($\rho_2 = 0.92$) & 222.2 & 231.6 \\
\hline
\end{tabular}
\end{table}

In table \ref{tab:var_etl_eur} we present VaR and ETL at the 5\% level.
The transition to a EUR-based operation significantly alters the firm's risk profile, moving from a single-commodity risk to a "double volatility" environment involving both copper prices and FX fluctuations.
In contrast to the USD results, where a strong pass-through provided a robust natural hedge, the EUR results in table \ref{tab:var_etl_eur} show a dramatic increase in VaR and ETL.
The greatly increased VaR reflects the fact that the firm is now exposed to the uncertainty of a weakening Euro.
Furthermore, the massive scale of these losses is amplified by the firm's increased nominal revenue, making the absolute impact of tail-risk events far more severe than in the original USD-denominated model.

\begin{table}[h!]
\centering
\caption{VaR and ETL at the 5\% level}
\label{tab:var_etl_eur}
\begin{tabular}{lcc}
\hline
\textbf{Scenario} & \textbf{VaR (5\%)} & \textbf{ETL (5\%)} \\
 & (EUR million) & (EUR million) \\
\hline
Fixed sales (no pass-through) & 15.4 & -6.6 \\
Partial pass-through ($\rho_1 = 0.74$) & 65.5 & 45.4 \\
Strong pass-through ($\rho_2 = 0.92$) & 146.6 & 135.8 \\
\hline
\end{tabular}
\end{table}



\section{Financial Risk and Combined Hedging Strategy}

In this stage of the risk analysis, we transition from EBIT to Net Income. This requires accounting for the firm's financing structure, specifically its 540 million EUR debt facility. This section evaluates the impact of interest rate volatility on the bottom line and demonstrates the benefits of a combined hedging strategy.

\subsection{Financing Structure and Interest Rate Exposure}
The firm carries 540 million EUR in debt, priced at a floating rate of 6-month SOFR plus a spread of 90 basis points (0.90\%). To model this exposure historically, we utilized the 6-month Treasury Bill rate as a proxy for the pre-2018 period when SOFR data was unavailable. 

The interest expense is calculated monthly as:
\[
    \text{Interest Expense}_{t} = \frac{\text{Debt} \cdot (\text{SOFR}_{t} + \text{Spread})}{12}
\]

This financing structure introduces a new source of volatility. Even if operating margins are protected via commodity hedges, a change in global interest rates can significantly influence Net Income.

\subsection{The Combined Hedge: Commodity and Interest Rates}
To achieve total risk stabilization, we propose a "Combined Hedge" strategy. In addition to the optimal copper futures hedge established before, the firm enters into an Interest Rate Swap (IRS). We assume a fixed swap rate of 4.5\%, which effectively transforms the floating-rate debt into a fixed-rate obligation of 5.4\% (Fixed Rate + Spread).

This removes the sensitivity of Net Income to the SOFR index, as the interest expense becomes a constant in the Profit and Loss (P\&L) statement:
\[
    \text{Net Income}_{Combined} = \text{EBIT}_{Hedged} - \overline{\text{Interest Expense}}_{Fixed}
\]

\subsection{Comparative Risk Analysis}
The results of the simulation demonstrate the effectiveness of fixing financial costs alongside operating costs. Table \ref{tab:combined_results} summarizes the Value at Risk (VaR) and Expected Tail Loss (ETL) for Net Income across all three sales regimes ($a$, $b_1$, $b_2$) at the 5\% level.

\subsection{Results and Risk Reduction}
Table \ref{tab:combined_risk} compares the risk profile of the commodity-hedged firm with and without interest rate protection.
The chosen scenario for the commodity is partial pass-through ($\rho = 0.74$).

\begin{table}[h!]
\centering
\caption{Net Income Risk Comparison (EUR millions)}
\label{tab:combined_risk}
\begin{tabular}{lccc}
\hline
\textbf{Strategy} & \textbf{VaR (5\%)} & \textbf{ETL (5\%)} \\ \hline
Hedged Commodity / Floating Debt & 53.1 & 30.4 \\
Combined Hedge (Commodity + IRS) & 53.4 & 30.8 \\ \hline
\end{tabular}
\end{table}

\subsection{Discussion of Findings}
In the partial pass-through regime, the risk profile is heavily influenced by the nominal scale of EUR-denominated operations. The simulation results show that:
\begin{itemize}
    \item The VaR remains relatively stable between the floating and combined strategies (approximately 53M EUR). This indicates that while the IRS stabilizes the interest expense, the dominant source of volatility in this regime remains the residual currency and commodity basis risk.
    \item The primary benefit of the IRS is the elimination of "interest rate shock" risk, which ensures that even during periods of tight monetary policy, the firm's debt service obligations remain constant.
\end{itemize}

% \subsection{Managerial Implications}
% The transition to a combined hedge represents the most robust risk management posture for the firm. While the commodity hedge protected the operating margin from copper price and FX volatility, the IRS eliminates the "financial tail risk" associated with tightening monetary policy. 

% As shown in the results, the Combined Hedge provides a higher floor for Net Income. By fixing the two largest sources of external uncertainty---commodity costs and financing costs---the firm can ensure debt serviceability and maintain stable dividends even in adverse market environments. This integrated approach to risk management is essential for an EU-based firm operating in volatile international markets.


% 9. naloga

\section{Extension to a Two-Commodity Exposure Framework}

This section extends the baseline analysis to a setting in which the firm is exposed to two traded commodities. At current prices, 57\% of the sales price is linked to copper, while an additional 15\% is linked to aluminium. Other variable material costs represent 5\% of sales, with all remaining profit and loss components unchanged relative to the initial specification.

\subsection{Commodity Price Dynamics and Cost Correlation}

Monthly spot price data are used for both commodities over the common sample period from November 2010 to June 2025. Material cost time series are constructed by scaling commodity prices such that their levels match the respective cost shares at current prices.

The correlation between copper and aluminium spot prices equals 0.886, indicating a strong positive co-movement. When expressed in terms of changes in material costs, the correlation remains high at 0.819. This implies that while some diversification benefits arise from the use of two commodities, overall material cost risk remains substantial and cannot be eliminated through diversification alone.

\subsection{EBIT Simulation Under Alternative Sales Assumptions}

EBIT is simulated under three alternative sales assumptions: (i) fixed sales, (ii) partial price pass-through with a sales--commodity correlation of 0.74, and (iii) a higher pass-through scenario with correlation 0.92. The point estimate of EBIT based on current prices amounts to USD 5.8 million and serves as a benchmark for the simulated distributions.

\subsection{Unhedged EBIT Risk in the Two-Commodity Case}

The unhedged EBIT distributions exhibit significant downside risk across all sales scenarios. At the 5\% confidence level, EBIT Value-at-Risk and Expected Tail Loss are particularly pronounced in the presence of sales pass-through. In the fixed-sales scenario, the 5\% VaR and ETL equal USD --47.7 million and USD --76.4 million, respectively. When sales are correlated with commodity prices, downside risk increases sharply, with VaR reaching USD --367 million for $\rho=0.74$ and USD --244 million for $\rho=0.92$. These results confirm that exposure to multiple correlated commodities constitutes a major source of earnings volatility.

\subsection{Two-Commodity Hedging Strategy}

To mitigate commodity-driven earnings risk, a minimum-variance hedging strategy is implemented using futures contracts on both copper and aluminium. In contrast to the single-commodity case, the optimal hedge now takes the form of a vector of hedge ratios,
\[
\mathbf{h}^* = \Sigma_{FF}^{-1} \Sigma_{F,CM},
\]
where $\Sigma_{FF}$ denotes the variance--covariance matrix of futures price changes and $\Sigma_{F,CM}$ their covariance with changes in total material costs.

The estimated hedge ratios equal
\[
h_{\text{Copper}} = 34{,}004 \quad \text{and} \quad h_{\text{Aluminium}} = 41.5,
\]
reflecting the dominant contribution of copper exposure to overall cost risk, consistent with its larger share in total material costs.

\subsection{Impact of Hedging on EBIT Risk}

The introduction of the two-instrument hedging strategy leads to a clear reduction in downside EBIT risk. In the fixed-sales scenario, the 5\% EBIT VaR improves from USD --47.7 million in the unhedged case to USD --53.4 million after hedging, while the corresponding ETL improves to USD --88.1 million. Hedging effects remain economically meaningful under both sales pass-through scenarios, although absolute risk levels remain higher due to the positive co-movement between revenues and commodity prices.

\subsection{Sensitivity to Derivative--Cost Correlation}

To assess the robustness of the hedging strategy, the analysis is repeated under alternative assumptions regarding the correlation between derivative instruments and material costs. As expected, higher correlations lead to stronger reductions in EBIT VaR and ETL, while lower correlations introduce basis risk and reduce hedge effectiveness. For instance, when the derivative--cost correlation increases from 0.60 to 0.75, EBIT VaR improves from USD --62.0 million to USD --50.0 million, accompanied by a corresponding improvement in ETL. These findings are consistent with the sensitivity results obtained in the single-commodity case.

\subsection{Discussion and Managerial Implications}

The two-commodity framework highlights the importance of jointly managing correlated commodity exposures. Although partial diversification arises from the use of multiple inputs, high cross-commodity correlation limits its effectiveness. A properly designed multi-instrument hedging strategy significantly reduces downside earnings risk, particularly when suitable derivative instruments with high correlation to underlying costs are available. From a managerial perspective, the results underscore that hedging decisions should be based on the joint distribution of commodity risks rather than on isolated single-commodity analyses.


\end{document}
